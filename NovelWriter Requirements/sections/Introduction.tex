
   
   
   
   - Provide an introductory section that briefly describes the purpose and scope of the software

also man kann halt mindmaps oder notizen aufschreiben und paar ordner mit bildern haben zum ideen sammeln

aber das ist halt nicht geordnet oder übersichtlich und alles an anderen Orten

wenn man das in einem hat, dabei irgendwie verlinkt hat und markieren kann auf gewisse weise, auch zeitlich einordnen, bisschen auf nem zeitstrahl events vorausplanen wenn man ideen hat und wenn man die dann umsetzt in kapitel das zuordnen und dann kann man später immer direkt nachschauen wann was war

oder überblick über die zeit in der Story haben, so ein event was irgendwann war kannst du nachgucken und direkt sagen "das war 3 jahre und 5 monate her" vor dem wo du bist

also so dass man auch für die geschichte sich ein zeitsystem ausdenken kann

jetzt nicht wie im weltraum mit mehreren zeitsystemen je nachdem auf welchem planet das ist zu komplex

aber man kann wie bei star wars sagen dass man zählt in jahren vor und nach der schlacht von javin oder sowas

oder wenn das fantasy ist und auf einem planeten 30h oder 5 tage nur sind pro woche dass man das eingeben kann und das berüchtigt wird

Und so Übersicht seiten mit allen characteren dass man niemanden vergisst und sich die relationships zwischen denen merken kann oder auch wie die sich verändern und man einfach direkt hat "wie hieß nochmal der zweite schulfreund oder die oma von dem und dem" wenn man dann bei buch 3 ist

da wäre generell ne suche praktisch wenn man ein wort benutzt und dann nachschlagen kann wo das vorkam und angezeigt bekommt was da an infos außenrum sind (wobei das eher für leser praktisch ist als den autor)

oder wenn man halt als Inspiration bilder gespeichert hat kann man die in clustern sortieren und dann kann man danach beschreiben wie bestimmte orte aussehen und hat die auch immer noch später

kann als überschrift dazu schreiben was es war und wenn man später nochmal von dem ort schreibt kann man durch die suche das direkt in dem cluster finden und sich anschauen was man da für vorstellungen hatte und ist wieder drin

oder du hast eine suche wo du alle vorkommen von bestimmten namen in dem buch nachschauen kannst, in wie vielen kapiteln jemand vorkam oder eher in welchem und dann kann man wieder nachschauen mit welchem character man den beschrieben hatte (oder macht das direkt auf die character steckbrief seite)